% ===================================================================
% CHAPTER 3: The EV2Gym Simulation Framework
% ===================================================================
\chapter{An Enhanced V2G Simulation Framework for Robust Control}
\label{chap:ev2gym}

Developing, validating, and benchmarking advanced control algorithms for Vehicle-to-Grid (V2G) systems is a complex endeavor. Real-world experimentation is often impractical due to prohibitive costs, logistical challenges, and risks to grid stability and vehicle hardware. To bridge the gap between theory and practice, a realistic, flexible, and standardized simulation environment is a scientific necessity. This thesis builds upon the foundation of \textbf{EV2Gym}, a state-of-the-art, open-source simulator designed for V2G smart charging research \footcite{orfanoudakis2024ev2gym}. This work, however, extends the original framework significantly, transforming it into a high-fidelity \textbf{digital twin} engineered not just for single-scenario optimization, but for the development and rigorous evaluation of \textbf{robust, generalist control agents}.

This enhanced framework offers a two-pronged approach to experimentation: it allows for deep-dive analysis of agents specialized for a single environment, while also introducing a novel methodology for training and testing agents designed to generalize across a multitude of diverse, unpredictable scenarios. This chapter provides an in-depth tour of this extended architecture, its data-driven models, and its unique evaluation capabilities, establishing the methodological bedrock for the rest of this work.

\section{Core Simulator Architecture}
The framework is built on the modular architecture of EV2Gym, which mirrors the key entities of a real-world V2G system. Its foundation on the OpenAI Gym (now Gymnasium) API is a cornerstone, providing a standardized agent-environment interface defined by the familiar language of states, actions, and rewards \footcite{brockman2016openai}.

\begin{figure}[H]
    \centering
   
    \includegraphics[width=0.8\linewidth]{Diagram_charge.png}
    \caption{Diagram of charging and discharging scheduling for EVs. \footcite{Dou}.}
    \label{fig:rl_cahrg}
\end{figure}
The architecture consists of several interacting components:
\begin{itemize}
    \item \textbf{Charge Point Operator (CPO):} The central intelligence of the simulation, managing the charging infrastructure and serving as the primary interface for the control algorithm (the DRL agent). The CPO aggregates system state information and dispatches control actions to individual chargers.
    \item \textbf{Chargers:} Digital representations of physical charging stations, configurable by type (AC/DC), maximum power, and efficiency. This allows for the simulation of heterogeneous charging infrastructures.
    \item \textbf{Power Transformers:} These components model the physical connection points to the grid, aggregating the electrical load from multiple chargers. Crucially, they enforce the physical power limits of the local distribution network and can model inflexible base loads (e.g., buildings) and local renewable generation (e.g., solar panels).
    \item \textbf{Electric Vehicles (EVs):} Dynamic and autonomous agents, each defined by its unique battery capacity, power limits, current and desired energy levels, and specific arrival and departure times.
\end{itemize}
The simulation process follows a reproducible three-phase structure: (1) \textbf{Initialization} from a comprehensive YAML configuration file, (2) a discrete-time \textbf{Simulation Loop} where the agent interacts with the environment, and (3) a final \textbf{Evaluation and Visualization} phase that generates standardized performance metrics.

\subsection{Software Implementation and Project Structure}
\label{sec:software_implementation}
While the conceptual architecture describes the simulator's components, the practical implementation is organized within a modular Python package named \texttt{ev2gym}. This structure promotes code reusability and a clear separation of concerns. The high-level experimentation scripts, such as \texttt{run\_experiments.py} and \texttt{train\_mpc\_approximator.py}, reside in the project's root directory and act as orchestrators, utilizing the core functionalities provided by the \texttt{ev2gym} package.

The key subdirectories within the \texttt{ev2gym} package are:
\begin{itemize}
    \item \texttt{baselines/}: This directory contains the implementations for all non-RL controllers. This includes rule-based heuristics (in \texttt{heuristics.py}) and, crucially, all variants of the Model Predictive Controllers (in \texttt{pulp\_mpc.py}).
    \item \texttt{rl\_agent/}: This is the central hub for all Reinforcement Learning logic. It contains modules for state vector construction (\texttt{state.py}), the library of available reward functions (\texttt{reward.py}), and the implementation of custom RL algorithms (\texttt{custom\_algorithms.py}).
    \item \texttt{utilities/}: A collection of helper functions and utility classes that are used across the entire framework.
    \item \texttt{models/}: This directory is designated for storing the serialized, pre-trained machine learning models, such as the Random Forest model used by the Approximate-Explicit MPC.
\end{itemize}
This modular software design allows for the independent development and testing of different components, such as control algorithms and reward functions, while maintaining a consistent and unified simulation environment.

\section{Core Physical Models}
The simulation's fidelity is anchored in its detailed, empirically validated models, which are essential for developing control strategies robust enough for real-world application.

\subsection{EV Model and Charging/Discharging Dynamics}
The framework implements a realistic two-stage charging/discharging model that captures the non-linear behavior of lithium-ion batteries, simulating both the \textbf{constant current (CC)} and \textbf{constant voltage (CV)} phases. Each EV is defined by a rich parameter set: maximum capacity ($E_{max}$), a minimum safety capacity ($E_{min}$), separate power limits for charging and discharging ($P_{ch}^{max}, P_{dis}^{max}$), and distinct efficiencies for each process ($\eta_{ch}, \eta_{dis}$).

\subsection{Battery Degradation Model}
To address the critical issue of battery health in V2G operations, the simulator incorporates a semi-empirical battery degradation model. It quantifies capacity loss ($Q_{lost}$) as the sum of two primary aging mechanisms \footcite{orfanoudakis2024ev2gym}: calendar aging and cyclic aging.

\begin{itemize}
    \item \textbf{Calendar Aging ($d_{cal}$):} Time-dependent capacity loss, influenced by the battery's average State of Charge (SoC) and temperature ($\Theta$). The formula is given by:
    \begin{equation}
        d_{cal} = 0.75 \cdot (\epsilon_0 \cdot \overline{SoC} - \epsilon_1) \cdot e^{-\epsilon_2/\Theta} \cdot \frac{t_{days}}{(t_{days}+1)^{0.25}}
    \end{equation}
    
    \item \textbf{Cyclic Aging ($d_{cyc}$):} Wear resulting from charge/discharge cycles, dependent on energy throughput ($E_{exchanged}$), depth-of-cycle (implicitly via $\overline{SoC}$), and the total accumulated charge ($Q_{acc}$). The formula is:
    \begin{equation}
        d_{cyc} = (\zeta_0 + \zeta_1 \cdot |\overline{SoC}-0.5|) \cdot \frac{E_{exchanged}}{\sqrt{Q_{acc}}}
    \end{equation}
\end{itemize}

The total capacity loss is the sum $Q_{lost} = d_{cal} + d_{cyc}$. This integrated model allows for the direct quantification of how different control strategies impact the battery's long-term State of Health (SoH), enabling the training of agents that balance profitability with battery preservation.

A key feature of this framework is that the physical parameters for this model ($\epsilon_0, \epsilon_1, \epsilon_2, \zeta_0, \zeta_1, Q_{acc}$) are not fixed. They can be empirically calibrated from real-world experimental data using the provided \texttt{Fit_battery.py} script, as detailed in Section \ref{sec:sim_architecture}.

\subsection{EV Behavior and Grid Models}
To ensure realism, the simulation is driven by authentic, open-source datasets. EV arrival/departure patterns and energy requirements are modeled using probability distributions derived from a large real-world dataset from \textbf{ElaadNL}. Grid conditions are similarly grounded in reality, using inflexible load data from the \textbf{Pecan Street} project and solar generation profiles from the \textbf{Renewables.ninja} platform \footcite{orfanoudakis2024ev2gym}.

\section{A Unified Experimentation and Evaluation Workflow}
A key contribution of this thesis is the development of a unified and powerful experimentation workflow, orchestrated by the main script \texttt{run\_experiments.py}. This script replaces the previous fragmented approach, providing a single, interactive interface to manage the entire lifecycle of training, benchmarking, and evaluation for V2G control agents. This workflow is designed to be both flexible for research and rigorous for evaluation, supporting the dual goals of developing specialized and generalized agents.

\subsection{Orchestration via \texttt{run\_experiments.py}}
The \texttt{run\_experiments.py} script acts as the central hub for all experimentation. It guides the user through an interactive command-line process, ensuring consistency and reproducibility. The key steps are:
\begin{enumerate}
    \item \textbf{Algorithm Selection:} The user can select from a predefined list of algorithms to benchmark. This includes Deep Reinforcement Learning agents (e.g., SAC, DDPG+PER, TQC), classical optimization methods (Model Predictive Control), and rule-based heuristics (e.g., Charge As Fast As Possible).
    \item \textbf{Scenario Selection:} The script automatically detects all available \texttt{.yaml} configuration files, allowing the user to choose one or more scenarios for the experiment. This choice determines the mode of operation (single-domain vs. multi-scenario).
    \item \textbf{Reward Function Selection:} The framework's flexibility is enhanced by allowing the user to dynamically select the reward function for the RL agents from the \texttt{reward.py} module.
    \item \textbf{Training and Benchmarking:} Based on the user's selections, the script proceeds to the optional training phase and then to a comprehensive benchmark, saving all results in a timestamped directory.
\end{enumerate}

\subsection{Dual-Mode Training: Specialists and Generalists}
The new workflow elegantly unifies the training of both "specialist" and "generalist" agents, a concept previously handled by separate scripts. The behavior is determined implicitly by the number of selected scenarios:
\begin{itemize}
    \item \textbf{Single-Domain Specialization:} If the user selects a single scenario, the script trains an RL agent exclusively on that environment. This produces a specialist agent, optimized to extract maximum performance from a specific, known set of conditions (e.g., a particular charging station topology and price profile).
    \item \textbf{Multi-Scenario Generalization:} If multiple scenarios are selected, the script automatically utilizes the \texttt{MultiScenarioEnv} wrapper. This custom Gymnasium environment dynamically switches between the different selected configurations at the start of each training episode. This process forces the agent to learn a robust and generalizable policy that performs well across a wide range of conditions, preventing overfitting to any single scenario. To handle the technical challenge of varying observation and action space sizes across scenarios, a \texttt{CompatibilityWrapper} is used to pad and slice the state-action vectors, enabling a single neural network policy to control heterogeneous environments.
\end{itemize}

\subsection{Reproducible Benchmarking and Evaluation}
To ensure a fair and scientifically valid comparison, the \texttt{run\_benchmark} function implements a rigorous evaluation protocol. For each scenario, it first generates a "replay" file containing the exact sequence of stochastic events (e.g., EV arrivals, energy demands). This exact same sequence is then used to evaluate every algorithm, eliminating randomness as a factor in performance differences. The script runs multiple simulations for statistical robustness, aggregates the mean results, and automatically generates a suite of comparative plots, including overall performance metrics and detailed battery degradation analyses.

\subsection{Interactive Web-Based Dashboard}
\label{sec:streamlit_app}
To complement the command-line-driven workflow, the project includes an interactive web-based dashboard built with the Streamlit library, executed via the \texttt{streamlit\_app.py} script. This graphical user interface (GUI) serves two primary functions, significantly enhancing usability and accessibility for experimentation and results analysis.

\subsubsection{Simulation Orchestrator}
The first part of the dashboard acts as a GUI wrapper for the \texttt{run\_experiments.py} script. It provides a user-friendly web form where users can:
\begin{itemize}
    \item Select which algorithms to benchmark from a multi-select list.
    \item Choose one or more scenarios to test.
    \item Pick a specific reward function for the RL agents from a dropdown menu.
    \item Set simulation parameters, such as the number of evaluation runs.
    \item Toggle optional steps, like running the \texttt{Fit\_battery.py} calibration or enabling RL model training.
\end{itemize}
Upon clicking the "Run Simulation" button, the application constructs the equivalent command-line arguments and executes \texttt{run\_experiments.py} as a subprocess. It captures and displays the console output in real-time on the web page, providing a seamless user experience without requiring direct terminal interaction.

\subsubsection{Results Visualizer}
The second part of the dashboard is a dedicated results browser. It automatically scans the \texttt{results/} directory and presents a list of all completed benchmark runs (organized by timestamp). The user can select a specific benchmark, and the application will find and display all the generated plots (e.g., performance comparisons, battery degradation graphs) directly on the page. This feature allows for quick and convenient inspection and comparison of outcomes from different experiments.

\section{Evaluation Metrics}
To ensure a fair and comprehensive comparison, all algorithms are evaluated against an identical set of pre-generated scenarios through a "replay" mechanism. The \textbf{mean} and \textbf{standard deviation} of performance are calculated across multiple simulation runs. The key metrics include:

\begin{itemize}
    \item \textbf{Total Profit (\$):} The net economic outcome, calculated as revenue from energy sales minus the cost of energy purchases.
    \[
    \Pi_{\text{total}} = \sum_{t=0}^{T_{\text{sim}}} \sum_{i=1}^{N} \left( C_{\text{sell}}(t) P_{\text{dis},i}(t) - C_{\text{buy}}(t) P_{\text{ch},i}(t) \right) \Delta t
    \]
    
    \item \textbf{Tracking Error (RMSE, kW):} For grid-balancing scenarios, this measures the root-mean-square error between the fleet's aggregated power and a target setpoint.
    \[
    E_{\text{track}} = \sqrt{\frac{1}{T_{\text{sim}}} \sum_{t=0}^{T_{\text{sim}}-1} \left( P_{\text{setpoint}}(t) - P_{\text{total}}(t) \right)^2}
    \]
    
    \item \textbf{User Satisfaction (Average):} The fraction of energy delivered compared to what was requested by the user, averaged across all EV sessions. A score of 1 indicates perfect service.
    \[
    US_{\text{avg}} = \frac{1}{N_{\text{EVs}}} \sum_{k=1}^{N_{\text{EVs}}} \min \left(1, \frac{E_k(t_k^{\text{dep}})}{E_k^{\text{des}}} \right)
    \]
    
    \item \textbf{Transformer Overload (kWh):} The total energy that exceeded the transformer's rated power limit. An ideal controller should achieve a value of 0.
    \[
    O_{\text{tr}} = \sum_{t=0}^{T_{\text{sim}}} \sum_{j=1}^{N_T} \max(0, P_j^{\text{tr}}(t) - P_j^{\text{tr,max}}) \cdot \Delta t
    \]
    
    \item \textbf{Battery Degradation (\$):} The estimated monetary cost of battery aging due to both cyclic and calendar effects.
    \[
    D_{\text{batt}} = \sum_{k=1}^{N_{\text{EVs}}} (\text{CyclicCost}_k + \text{CalendarCost}_k)
    \]
\end{itemize}

\section{Simulator Implementation Details}
\label{sec:sim_architecture}
During the analysis and implementation of new metrics, fundamental details about the \texttt{EV2Gym} simulator's architecture emerged, which warrant documentation. The configuration of Electric Vehicles (EVs) and the calculation of their degradation follow a specific logic dependent on a key parameter in the \texttt{.yaml} configuration files.

\subsubsection{Vehicle Definition Modes}
The simulator operates in two distinct modes, controlled by the boolean flag \texttt{heterogeneous\_ev\_specs}:
\begin{itemize}
    \item \textbf{Heterogeneous Mode (\texttt{True}):} In this mode, the simulator ignores the default vehicle specifications in the \texttt{.yaml} file. Instead, it loads a list of vehicle profiles from an external JSON file, specified by the \texttt{ev\_specs\_file} parameter (e.g., \texttt{ev\_specs\_v2g\_enabled2024.json}). This allows for the creation of a realistic fleet with diverse battery capacities, charging powers, and efficiencies. For instance, the fleet may include a \textbf{Peugeot 208} with a 46.3 kWh battery and a 7.4 kW charge rate, alongside a \textbf{Volkswagen ID.4} with a 77 kWh battery and an 11 kW charge rate. A vehicle is randomly selected from this list for each new arrival event.
    \item \textbf{Homogeneous Mode (\texttt{False}):} In this mode, the external JSON file is ignored. All vehicles created in the simulation are identical, and their characteristics are defined exclusively by the \texttt{ev:} block within the \texttt{.yaml} configuration file. The \texttt{battery\_capacity} parameter in this block becomes the single source of truth for the entire fleet.
\end{itemize}

\subsubsection{Empirical Calibration of the Degradation Model}
A significant enhancement in this work is the move towards a more physically representative and flexible battery degradation model. While the underlying semi-empirical model for calendar and cyclic aging remains, the methodology for parameterizing it has been fundamentally improved, addressing previous inconsistencies.

This is achieved through the \texttt{Fit\_battery.py} script, a new utility for empirical model calibration. The script implements the following workflow:
\begin{enumerate}
    \item \textbf{Data Loading:} It loads time-series data from real-world battery aging experiments. The expected data includes measurements of capacity loss over time, along with contextual variables like state of charge (SoC), temperature, and energy throughput.
    \item \textbf{Model Fitting:} Using the \texttt{curve\_fit} function from the SciPy library, the script fits the parameters of the \texttt{Qlost\_model} (which combines calendar and cyclic aging) to the empirical data. This optimization process finds the physical constants (e.g., $\epsilon_0, \zeta_0$) that best explain the observed degradation.
    \item \textbf{Parameter Export:} The script outputs the calibrated parameters. These values can then be used directly in the simulator's configuration, ensuring that the degradation model for a specific EV fleet is grounded in experimental evidence for that battery type.
\end{enumerate}
This calibration workflow, integrated optionally into the main \texttt{run\_experiments.py} script, elevates the simulation's fidelity. It allows the framework to move beyond a single, fixed degradation model (previously calibrated for a 78 kWh battery) and enables the creation of high-fidelity digital twins for a wide variety of EV batteries, provided that the necessary experimental data is available.

\section{Reinforcement Learning Formulation}
The control problem is formalized as a Markov Decision Process (MDP), defined by the tuple $(S, A, P, R, \gamma)$.

\subsection{State Space ($S$)}
The state $s_t \in S$ is a feature vector providing a snapshot of the environment at time $t$. A representative state, as defined in modules like \texttt{V2G\_profit\_max\_loads.py}, includes:
\[
 s_t = [t, P_{\text{total}}(t-1), \mathbf{c}(t, H), \mathbf{L}_1(t, H), \mathbf{PV}_1(t, H), \dots, \mathbf{s}^{\text{EV}}_1(t), \dots, \mathbf{s}^{\text{EV}}_N(t)]^T
\]
where the components are:
\begin{itemize}
    \item $t$: The current time step.
    \item $P_{\text{total}}(t-1)$: The aggregated power from the previous time step.
    \item $\mathbf{c}(t, H)$: A vector of \textbf{predicted future} electricity prices over a horizon $H$.
    \item $\mathbf{L}_j(t, H), \mathbf{PV}_j(t, H)$: Forecasts for inflexible loads and solar generation.
    \item $\mathbf{s}^{\text{EV}}_i(t) = [\text{SoC}_i(t), t^{\text{dep}}_i - t]$: Key information for each EV $i$, including its State of Charge and remaining time until departure.
\end{itemize}

\subsection{Action Space ($A$)}
The action $a_t \in A$ is a continuous vector in $\mathbb{R}^N$, where $N$ is the number of chargers. For each charger $i$, the command $a_i(t) \in [-1, 1]$ is a normalized value that is translated into a power command:
\begin{itemize}
    \item If $a_i(t) > 0$, the EV is charging: $P_i(t) = a_i(t) \cdot P^{\text{max}}_{\text{charge}, i}$.
    \item If $a_i(t) < 0$, the EV is discharging (V2G): $P_i(t) = a_i(t) \cdot P^{\text{max}}_{\text{discharge}, i}$.
\end{itemize}

\subsection{Reward Function}
The reward function $R(t)$ encodes the objectives of the control agent. The framework allows for the selection of different reward functions from the \texttt{reward.py} module to suit various goals. Key examples include:
\begin{itemize}
    \item \textbf{Profit Maximization with Penalties} (\texttt{ProfitMax\_TrPenalty\_UserIncentives}): This function creates a balance between economic gain and physical constraints.
    \[
    R(t) = \underbrace{\text{Profit}(t)}_{\text{Economic Gain}} - \underbrace{\lambda_1 \cdot \text{Overload}(t)}_{\text{Grid Penalty}} - \underbrace{\lambda_2 \cdot \text{Unsatisfaction}(t)}_{\text{User Penalty}}
    \]
    The agent is rewarded for profit but penalized for overloading transformers and for failing to meet the charging needs of departing drivers.
    
    \item \textbf{Squared Tracking Error} (\texttt{SquaredTrackingErrorReward}): Used for grid service applications where precision is paramount.
    \[
    R(t) = - \left( P_{\text{setpoint}}(t) - \sum_{i=1}^N P_i(t) \right)^2
    \]
    The reward is the negative squared error from the power setpoint, incentivizing the agent to minimize this error at all times.
\end{itemize}

By using this enhanced framework, this thesis moves beyond single-scenario optimization to develop and validate an intelligent V2G control agent that is not only high-performing but also robust, adaptable, and ready for the complexities of real-world deployment.

\section{Reinforcement Learning Algorithms}
This work benchmarks several state-of-the-art Deep Reinforcement Learning algorithms. The following sections provide a detailed mathematical description of the selected off-policy, actor-critic algorithms.

\subsubsection{Soft Actor-Critic (SAC)}
SAC is an off-policy actor-critic algorithm designed for continuous action spaces that optimizes a stochastic policy. Its core feature is entropy maximization, which encourages exploration and improves robustness. The agent aims to maximize not only the expected sum of rewards but also the entropy of its policy.
\paragraph{Soft Actor-Critic (SAC)}
\begin{figure}[H]
    \centering
    \includegraphics[width=0.5\linewidth]{SAC.png}
    \caption{SAC Structure Image from \footcite{Qiu2023}}}
    \label{fig:SAC}
\end{figure}

The objective function is:
\[
J(\pi) = \sum_{t=0}^{T} \mathbb{E}_{(s_t, a_t) \sim \rho_\pi} \left[ r(s_t, a_t) + \alpha \mathcal{H}(\pi(\cdot|s_t)) \right]
\]
where $\mathcal{H}$ is the entropy of the policy $\pi$ and $\alpha$ is the temperature parameter, which controls the trade-off between reward and entropy.

\paragraph{Implementation Details}
The implementation of SAC is built upon the robust, industry-standard \textbf{Stable-Baselines3} library, which provides a highly optimized and well-tested version of the algorithm. The standard \texttt{SAC} class from this library is used directly, leveraging its PyTorch-based backend for efficient training and inference.

SAC uses a soft Q-function, trained to minimize the soft Bellman residual:
\[
L(\theta_Q) = \mathbb{E}_{(s_t, a_t, r_t, s_{t+1}) \sim D} \left[ \left( Q(s_t, a_t) - \left(r_t + \gamma V_{\bar{\psi}}(s_{t+1})\right) \right)^2 \right]
\]
where $D$ is the replay buffer and the soft state value function $V$ is defined as:
\[
V_{\text{soft}}(s_t) = \mathbb{E}_{a_t \sim \pi} [Q_{\text{soft}}(s_t, a_t) - \alpha \log \pi(a_t|s_t)]
\]
To mitigate positive bias, SAC employs two Q-networks (Clipped Double-Q) and takes the minimum of the two target Q-values during the Bellman update.

\subsubsection{Deep Deterministic Policy Gradient + PER (DDPG+PER)}
DDPG is an off-policy algorithm that concurrently learns a deterministic policy $\mu(s | \theta^\mu)$ and a Q-function $Q(s, a | \theta^Q)$. It is the deep-learning extension of the DPG algorithm for continuous action spaces.

\begin{figure}[H]
    \centering
    \includegraphics[width=0.5\linewidth]{DDPG.png}
    \caption{DDPG Structure (Image from \footcite{Qiu2023}}
    \label{fig:DDPG}
\end{figure}
\begin{itemize}
    \item \textbf{Critic Update:} The critic is updated by minimizing the mean-squared Bellman error, similar to Q-learning. Target networks ($Q'$ and $\mu'$) are used to stabilize training.
    \[
    L(\theta^Q) = \mathbb{E}_{(s_t, a_t, r_t, s_{t+1}) \sim D} \left[ (y_t - Q(s_t, a_t | \theta^Q))^2 \right]
    \]
    where the target $y_t$ is given by:
    \[
    y_t = r_t + \gamma Q'(s_{t+1}, \mu'(s_{t+1}|
    \theta^{\mu'})|
    \theta^{Q'})
    \]
    \item \textbf{Actor Update:} The actor is updated using the deterministic policy gradient theorem:
    \[
    \nabla_{\theta^\mu} J \approx \mathbb{E}_{s_t \sim D} [\nabla_a Q(s, a | \theta^Q)|_{s=s_t, a=\mu(s_t)} \nabla_{\theta^\mu} \mu(s_t | \theta^\mu)]
    \]
    \item \textbf{Prioritized Experience Replay (PER):} This work enhances DDPG with PER. Instead of uniform sampling from the replay buffer $D$, PER samples transitions based on their TD-error, prioritizing those where the model has the most to learn. The probability of sampling transition $i$ is:
    \[
    P(i) = \frac{p_i^\beta}{\sum_k p_k^\beta}
    \]
    where $p_i = |\delta_i| + \epsilon$ is the priority based on the TD-error $\delta_i$, and $\beta$ controls the degree of prioritization. To correct for the bias introduced by non-uniform sampling, PER uses importance-sampling (IS) weights.
\end{itemize}

\paragraph{Implementation Details}
To integrate Prioritized Experience Replay (PER) with DDPG, a custom class, \texttt{CustomDDPG}, was developed in the \texttt{ev2gym/rl\_agent/custom\_algorithms.py} module. This class inherits from the standard \texttt{DDPG} agent provided by \textbf{Stable-Baselines3}. The core \texttt{train} method is overridden to replace the default uniform-sampling replay buffer with one that supports prioritized sampling. This involves calculating TD-errors for each transition, updating their priorities in the buffer, and using the resulting importance-sampling weights during the critic update, thereby focusing the learning process on the most informative experiences.

\subsubsection{Truncated Quantile Critics (TQC)}
TQC enhances the stability of SAC by modeling the entire distribution of returns instead of just its mean. This is achieved through quantile regression and a novel truncation mechanism to combat Q-value overestimation.

\begin{itemize}
    \item \textbf{Distributional Learning:} TQC employs a set of $N$ critic networks, \{$Q_{\phi_i}(s, a)\}_{i=1}^{N}$, each trained to estimate a specific quantile $\tau_i$ of the return distribution. The target quantiles are implicitly defined as $\tau_i = \frac{i-0.5}{N}$. The critics are trained by minimizing the quantile Huber loss, $L_{QH}$.
    
    \item \textbf{Distributional Target Calculation:} A distributional target is constructed for the Bellman update. First, an action is sampled from the target policy for the next state: $\tilde{a}_{t+1} \sim \pi_{\theta'}(\cdot|s_{t+1})$. Then, a set of $N$ Q-value estimates for the next state is obtained from the $N$ target critic networks: \{$Q_{\phi'_j}(s_{t+1}, \tilde{a}_{t+1})\}_{j=1}^{N}$.
    
    \item \textbf{Truncation:} This is the key idea of TQC. To combat overestimation, the algorithm discards the $k$ largest Q-value estimates from the set of $N$ target values. This truncation removes the most optimistic estimates, which are a primary source of bias, leading to more conservative and stable updates.
    
    \item \textbf{Critic Update:} The target value for updating the $i$-th critic is formed using the Bellman equation with the truncated set of next-state Q-values. The overall critic loss is the sum of the quantile losses across all critics:
    \[
    L(\phi) = \sum_{i=1}^{N} \mathbb{E}_{(s,a,r,s') \sim D} \left[ L_{QH}\left(r + \gamma Q_{\text{trunc}}(s', \tilde{a}') - Q_{\phi_i}(s,a) \right) \right]
    \]
    where $Q_{\text{trunc}}$ represents the value derived from the truncated set of target quantiles.
\end{itemize}

\paragraph{Implementation Details}
The TQC algorithm is leveraged from the \textbf{SB3-Contrib} library, a collection of community-contributed extensions to Stable-Baselines3. Using the library's standard \texttt{TQC} implementation allows the framework to benefit from this state-of-the-art distributional RL algorithm without requiring a custom implementation from scratch.



\section{Reinforcement Learning Algorithms}
This work benchmarks several state-of-the-art Deep Reinforcement Learning algorithms. The following sections provide a detailed mathematical description of the selected off-policy, actor-critic algorithms that form the core of the experimental comparison.
\noindent

\subsubsection{Soft Actor-Critic (SAC)}
SAC is an off-policy actor-critic algorithm designed for continuous action spaces that optimizes a stochastic policy. Its distinguishing feature is the explicit incorporation of entropy maximization into the objective function, which promotes exploration and yields policies that are more robust to model inaccuracies and environmental perturbations. Unlike traditional RL algorithms that focus solely on reward maximization, SAC aims to maximize both the expected cumulative reward and the entropy of the policy distribution.
\noindent

\begin{figure}[H]
    \centering
    \includegraphics[width=0.5\linewidth]{SAC.png}
    \caption{SAC actor-critic architecture. The actor network outputs a stochastic policy parameterized by a mean and standard deviation, while two separate Q-networks (critics) estimate state-action values. The minimum of the two Q-values is used for policy updates to mitigate overestimation bias. Image from \footcite{Qiu2023}.}
    \label{fig:SAC}
\end{figure}

The objective function is:
\[
J(\pi) = \sum_{t=0}^{T} \mathbb{E}_{(s_t, a_t) \sim \rho_\pi} \left[ r(s_t, a_t) + \alpha \mathcal{H}(\pi(\cdot|s_t)) \right]
\]
\noindent
where $\mathcal{H}$ is the entropy of the policy $\pi$, defined as $\mathcal{H}(\pi(\cdot|s_t)) = -\mathbb{E}_{a_t \sim \pi}[\log \pi(a_t|s_t)]$, and $\alpha$ is the temperature parameter, which controls the trade-off between reward exploitation and entropy-driven exploration. A higher $\alpha$ encourages more stochastic behavior, while a lower $\alpha$ biases the policy toward deterministic, reward-maximizing actions.
\noindent

\paragraph{Implementation Details}
The implementation of SAC is built upon the robust, industry-standard \textbf{Stable-Baselines3} library, which provides a highly optimized and well-tested version of the algorithm. The standard \texttt{SAC} class from this library is used directly, leveraging its PyTorch-based backend for efficient training and inference.
\noindent

SAC employs a soft Q-function, which incorporates the policy entropy into the value estimation. The soft Q-function is trained to minimize the soft Bellman residual:
\[
L(\theta_Q) = \mathbb{E}_{(s_t, a_t, r_t, s_{t+1}) \sim D} \left[ \left( Q(s_t, a_t) - \left(r_t + \gamma V_{\bar{\psi}}(s_{t+1})\right) \right)^2 \right]
\]
\noindent
where $D$ is the replay buffer, $\gamma$ is the discount factor, and the soft state value function $V$ is defined as:
\[
V_{\text{soft}}(s_t) = \mathbb{E}_{a_t \sim \pi} [Q_{\text{soft}}(s_t, a_t) - \alpha \log \pi(a_t|s_t)]
\]
\noindent
To mitigate the positive bias inherent in Q-learning, SAC employs two independent Q-networks (implementing the Clipped Double-Q technique) and uses the minimum of the two target Q-values during the Bellman update. This conservative approach prevents overestimation from propagating through the learning process. Additionally, SAC can automatically tune the temperature parameter $\alpha$ during training by treating it as a constrained optimization problem, ensuring that the policy entropy remains above a target threshold.
\noindent

\subsubsection{Deep Deterministic Policy Gradient + PER (DDPG+PER)}
DDPG is an off-policy algorithm that concurrently learns a deterministic policy $\mu(s | \theta^\mu)$ and a Q-function $Q(s, a | \theta^Q)$. It extends the Deterministic Policy Gradient (DPG) theorem to deep neural networks, enabling continuous control in high-dimensional state spaces. Unlike stochastic policy methods, DDPG directly outputs a single action for each state, which can be advantageous in environments where deterministic behavior is preferred or when the action space is high-dimensional.
\noindent

\begin{figure}[H]
    \centering
    \includegraphics[width=0.5\linewidth]{DDPG.png}
    \caption{DDPG actor-critic architecture. The actor network produces a deterministic action, which is evaluated by the Q-network (critic). Both networks maintain slowly-updating target networks ($\mu'$ and $Q'$) for stable Bellman updates. During training, exploration noise is added to the actor's output. Image from \footcite{Qiu2023}.}
    \label{fig:DDPG}
\end{figure}

\begin{itemize}
    \item \textbf{Critic Update:} The critic is updated by minimizing the mean-squared Bellman error, analogous to Q-learning. Target networks ($Q'$ and $\mu'$) are used to stabilize training by providing consistent targets during the temporal difference update.
    \[
    L(\theta^Q) = \mathbb{E}_{(s_t, a_t, r_t, s_{t+1}) \sim D} \left[ (y_t - Q(s_t, a_t | \theta^Q))^2 \right]
    \]
    \noindent
    where the target $y_t$ is given by:
    \[
    y_t = r_t + \gamma Q'(s_{t+1}, \mu'(s_{t+1}|\theta^{\mu'})|\theta^{Q'})
    \]
    \noindent
    The target networks are updated via soft updates (polyak averaging): $\theta' \leftarrow \tau \theta + (1-\tau)\theta'$, where $\tau \ll 1$ is a small constant.
    \noindent
    
    \item \textbf{Actor Update:} The actor is updated using the deterministic policy gradient theorem, which states that the gradient of the expected return with respect to the policy parameters can be computed as:
    \[
    \nabla_{\theta^\mu} J \approx \mathbb{E}_{s_t \sim D} [\nabla_a Q(s, a | \theta^Q)|_{s=s_t, a=\mu(s_t)} \nabla_{\theta^\mu} \mu(s_t | \theta^\mu)]
    \]
    \noindent
    This gradient ascent step improves the policy by moving it in the direction that increases the Q-value according to the critic's estimate.
    \noindent
    
    \item \textbf{Prioritized Experience Replay (PER):} This work enhances DDPG with PER, which addresses a fundamental limitation of uniform experience replay. Instead of sampling transitions uniformly from the replay buffer $D$, PER samples transitions according to their temporal-difference (TD) error, prioritizing experiences where the model's predictions are most inaccurate and therefore most informative. The probability of sampling transition $i$ is:
    \[
    P(i) = \frac{p_i^\beta}{\sum_k p_k^\beta}
    \]
    \noindent
    where $p_i = |\delta_i| + \epsilon$ is the priority based on the TD-error $\delta_i = y_t - Q(s_t, a_t)$, $\epsilon$ is a small constant to ensure non-zero probabilities, and $\beta$ controls the degree of prioritization. To correct for the bias introduced by non-uniform sampling, PER applies importance-sampling (IS) weights during the gradient update:
    \[
    w_i = \left(\frac{1}{N \cdot P(i)}\right)^\alpha
    \]
    \noindent
    where $\alpha$ controls the amount of bias correction. These weights ensure that the gradient estimator remains unbiased despite the non-uniform sampling distribution.
    \noindent
\end{itemize}

\paragraph{Implementation Details}
To integrate Prioritized Experience Replay (PER) with DDPG, a custom class, \texttt{CustomDDPG}, was developed in the \\ \noindent \texttt{ev2gym/rl\_agent/custom\_algorithms.py} module. This class inherits from the standard \texttt{DDPG} agent provided by \textbf{Stable-Baselines3}. The core \texttt{train} method is overridden to replace the default uniform-sampling replay buffer with one that supports prioritized sampling. This involves calculating TD-errors for each transition, updating their priorities in the buffer, and using the resulting importance-sampling weights during the critic update, thereby focusing the learning process on the most informative experiences.
\noindent

\subsubsection{Truncated Quantile Critics (TQC)}
TQC represents a significant advancement over standard SAC by modeling the entire distribution of returns rather than just its expected value. This distributional perspective, combined with a novel truncation mechanism, provides superior protection against Q-value overestimation—a persistent pathology in off-policy RL that can lead to catastrophic policy degradation.
\noindent

\begin{itemize}
    \item \textbf{Distributional Learning:} TQC employs an ensemble of $N$ critic networks, \{$Q_{\phi_i}(s, a)\}_{i=1}^{N}$, where each network is trained to estimate a specific quantile $\tau_i$ of the return distribution. The target quantiles are implicitly defined as $\tau_i = \frac{i-0.5}{N}$, uniformly covering the cumulative distribution function. The critics are trained by minimizing the quantile Huber loss, $L_{QH}$, which is a robust asymmetric loss function that penalizes overestimation and underestimation differently:
    \[
    L_{QH}(\delta) = |\tau_i - \mathbb{I}_{\{\delta < 0\}}| \cdot L_\kappa(\delta)
    \]
    \noindent
    where $L_\kappa$ is the Huber loss with threshold $\kappa$, and $\delta$ is the temporal difference error.
    \noindent
    
    \item \textbf{Distributional Target Calculation:} A distributional target is constructed for the Bellman update. First, an action is sampled from the target policy for the next state: $\tilde{a}_{t+1} \sim \pi_{\theta'}(\cdot|s_{t+1})$. Then, a set of $N$ Q-value estimates for the next state is obtained from the $N$ target critic networks: \{$Q_{\phi'_j}(s_{t+1}, \tilde{a}_{t+1})\}_{j=1}^{N}$. This ensemble provides a rich representation of the value distribution's uncertainty.
    \noindent
    
    \item \textbf{Truncation:} This is the central innovation of TQC. To combat overestimation, the algorithm discards the $k$ largest Q-value estimates from the set of $N$ target values. This truncation removes the most optimistic (and typically most biased) estimates, which are the primary drivers of overestimation in ensemble methods. By dropping these outliers, TQC obtains more conservative and stable target values. The remaining $(N-k)$ quantile estimates are then used to compute the Bellman target, effectively implementing a form of pessimism that counteracts the inherent optimism bias in temporal-difference learning.
    \noindent
    
    \item \textbf{Critic Update:} The target value for updating the $i$-th critic is formed using the Bellman equation with the truncated set of next-state Q-values. The overall critic loss is the sum of the quantile losses across all critics:
    \[
    L(\phi) = \sum_{i=1}^{N} \mathbb{E}_{(s,a,r,s') \sim D} \left[ L_{QH}\left(r + \gamma Q_{\text{trunc}}(s', \tilde{a}') - Q_{\phi_i}(s,a) \right) \right]
    \]
    \noindent
    where $Q_{\text{trunc}}$ represents the value derived from the truncated set of target quantiles. The actor is then updated using the mean of all (non-truncated) critic estimates, similar to standard SAC, but benefiting from the improved stability of the distributional value estimates.
    \noindent
\end{itemize}

\paragraph{Implementation Details}
The TQC algorithm is leveraged from the \textbf{SB3-Contrib} library, a collection of community-contributed extensions to Stable-Baselines3. Using the library's standard \texttt{TQC} implementation allows the framework to benefit from this state-of-the-art distributional RL algorithm without requiring a custom implementation from scratch. The implementation maintains the entropy-regularized objective of SAC while incorporating the distributional critics and truncation mechanism that define TQC's superior performance characteristics.
\noindent
%%%%%%%%%%%%%%%%%%%%%%%%%%%%%%

\subsection{A History-Based Adaptive Reward for Profit Maximization}
\label{sec:adaptive_reward}

\paragraph{Implementation Details}
This reward function, along with a suite of other reward strategies, is implemented in the
\\
\noindent
 \texttt{ev2gym/rl\_agent/reward.py} module. The main experimentation script, \texttt{run\_experiments.py}, allows the user to dynamically select which reward function to use for a given training run. The \texttt{FastProfitAdaptiveReward} function, specifically, uses \texttt{collections.deque} objects attached to the environment instance to efficiently maintain a sliding window of recent performance for satisfaction and overload events.
\noindent
To steer the learning agent towards policies that are both highly profitable and operationally reliable, we have designed a novel, history-based adaptive reward function named \texttt{FastProfitAdaptiveReward}. This function challenges traditional static-weight penalty approaches by introducing a dynamic feedback mechanism where penalty severity responds directly to the agent's recent performance. The underlying principle is straightforward but powerful: prioritize economic profit aggressively, while employing adaptive penalties as guardrails that tighten only when the agent begins to systematically violate operational constraints.
\noindent
The total reward at each timestep $t$, $R_t$, is calculated as the net economic profit minus any active penalties for user dissatisfaction or transformer overload.
\noindent

\begin{equation}
    R_t = \Pi_t - P_t^{\text{sat}} - P_t^{\text{tr}}
\end{equation}

\subsubsection{Economic Profit}
The foundation of the reward signal is the direct, instantaneous economic profit, $\Pi_t$. This component provides an unambiguous incentive for the agent to exploit market dynamics, encouraging charging during low-price periods and discharging (V2G) during high-price periods.
\noindent

\begin{equation}
    \Pi_t = \sum_{i=1}^{N} \left( C_t^{\text{sell}} \cdot P_{i,t}^{\text{dis}} - C_t^{\text{buy}} \cdot P_{i,t}^{\text{ch}} \right) \Delta t
\end{equation}
\noindent
where $N$ is the number of connected EVs, $C_t^{\text{sell}}$ and $C_t^{\text{buy}}$ are the electricity prices, and $P_{i,t}^{\text{dis}}$ and $P_{i,t}^{\text{ch}}$ are the discharging and charging powers for EV $i$.
\noindent

\subsubsection{Adaptive User Satisfaction Penalty}
The penalty for failing to meet user charging demands, $P_t^{\text{sat}}$, departs from fixed-value approaches. It adapts based on the system's recent performance history. The environment maintains a short-term memory of the average user satisfaction over the last 100 timesteps, from which we calculate an average satisfaction score, $\bar{S}_{hist}$.
\noindent
A \textit{satisfaction severity multiplier}, $\lambda_t^{\text{sat}}$, is then calculated. This multiplier grows quadratically as the historical average satisfaction drops. When the system has been performing poorly, the consequences for a new failure become substantially more severe.
\noindent
\begin{equation}
    \lambda_t^{\text{sat}} = \lambda_{\text{base}}^{\text{sat}} \cdot (1 - \bar{S}_{hist})^2
\end{equation}
\noindent
where $\lambda_{\text{base}}^{\text{sat}}$ is a base scaling factor (e.g., 20.0). A penalty is only applied if any departing EV's satisfaction, $S_k$, falls below a critical threshold (e.g., 95\%). The magnitude of the penalty is the product of the adaptive multiplier and the current satisfaction deficit.
\noindentW
\begin{equation}
    P_t^{\text{sat}} = \lambda_t^{\text{sat}} \cdot (1 - \min(S_k)) \quad \forall k \in \text{EVs departing at } t
\end{equation}
\noindent

This establishes a feedback loop with non-trivial consequences: a single, isolated failure in an otherwise well-performing system results in a mild penalty, whereas persistent failures trigger rapidly escalating penalties that force behavioral correction.
\noindent

\subsubsection{Adaptive Transformer Overload Penalty}
The transformer overload penalty, $P_t^{\text{tr}}$, operates on a similar adaptive principle, responding to the recent frequency of overloads. The environment tracks how often an overload has occurred in the last 100 timesteps, yielding an overload frequency, $F_{hist}^{\text{tr}}$.
\noindent

This frequency drives the computation of a linear \textit{overload severity multiplier}, $\lambda_t^{\text{tr}}$. Higher overload frequency translates directly to higher penalties for new violations.
\noindent

\begin{equation}
    \lambda_t^{\text{tr}} = \lambda_{\text{base}}^{\text{tr}} \cdot F_{hist}^{\text{tr}}
\end{equation}
\noindent
where $\lambda_{\text{base}}^{\text{tr}}$ is a base scaler (e.g., 50.0). If the total power drawn, $P_j^{\text{total}}(t)$, exceeds the transformer's limit, $P_j^{\text{max}}$, a penalty is applied. This penalty consists of a small, fixed base amount plus the adaptive component, which scales with the magnitude of the current overload.
\noindent

\begin{equation}
    P_t^{\text{tr}} = P_{\text{base}} + \lambda_t^{\text{tr}} \cdot \sum_{j=1}^{N_T} \max(0, P_j^{\text{total}}(t) - P_j^{\text{max}})
\end{equation}
\noindent

This mechanism enforces a critical lesson: while an occasional, minor overload might be tolerable in pursuit of high profit, habitual overloading becomes increasingly unsustainable as penalties escalate.
\noindent

\subsubsection{Rationale and Significance}
This history-based adaptive reward function represents a departure from static or purely state-based approaches. By making penalty weights a function of recent performance history, we provide a more sophisticated learning signal. The agent is not excessively punished for isolated, exploratory actions that might lead to minor constraint violations. Instead, it is strongly discouraged from developing policies that produce chronic system failures.
\noindent
The approach mirrors realistic management objectives: maintain high performance on average, and react decisively only when performance trends begin to deteriorate. This method is also computationally efficient, avoiding complex state-dependent calculations in favor of simple updates to historical data queues. The reward structure guides the agent to discover policies that balance economic objectives with operational reliability, achieving an intelligent equilibrium between profit ambition and system safety constraints.
\noindent


\section{Online MPC Formulation (PuLP Implementation)}

The Model Predictive Control (MPC) implemented with PuLP solves a profit maximization problem at each control interval over a finite prediction horizon $H$. This formulation is designed for online, real-time control, where decisions are made based on the current system state and future predictions. This approach is often termed an "implicit" MPC because the control law is not pre-computed; instead, it is found implicitly by solving a full optimization problem at every control interval. The logic is formally detailed in Algorithm \ref{alg:milp_mpc}.

\paragraph{Implementation Details}
This online controller is implemented as the \texttt{OnlineMPC\_Solver} class within the \texttt{ev2gym/baselines/pulp\_mpc.py} module. At each invocation of its \texttt{get\_action} method, it dynamically constructs the full Mixed-Integer Linear Program (MILP) described below using the \textbf{PuLP} modeling library. PuLP acts as a high-level modeling interface, which then calls an underlying solver. The problem is then solved using the default CBC (COIN-OR Branch and Cut) solver, an open-source solver capable of handling MILPs.

\begin{algorithm}[H]
\caption{Online Model Predictive Control (MILP)}
\label{alg:milp_mpc}
\begin{algorithmic}[1]
\Function{GetAction}{$E_{\text{initial}}, k_{\text{dep}}, c_{\text{buy}}, c_{\text{sell}}, H, N_c$}
    \State \textbf{Inputs:} Current energy states $E_{\text{initial}}$, departure times $k_{\text{dep}}$, price vectors $c_{\text{buy}}, c_{\text{sell}}$, prediction horizon $H$, control horizon $N_c$.
    
    \State \textbf{Initialize Problem:} Create a new MILP problem for profit maximization.
    
    \State \textbf{Define Variables} for $i \in \text{CS}, k \in [0, H-1]$:
    \State $P^{\text{ch}}_{i,k}, P^{\text{dis}}_{i,k} \in \mathbb{R}^+$ \Comment{Continuous power variables}
    \State $E_{i,k} \in \mathbb{R}^+$ \Comment{Continuous energy state variables}
    \State $z_{i,k} \in \{0, 1\}$ \Comment{Binary charge/discharge mode variables}
    
    \State \textbf{Define Objective Function:}
    \State Profit $\leftarrow \sum_{k=0}^{H-1} \sum_{i \in \text{CS}} \left[ (c^{\text{user}} - c^{\text{buy}}_k - c^{\text{deg}}) P^{\text{ch}}_{i,k} + (c^{\text{sell}}_k - c^{\text{deg}}) P^{\text{dis}}_{i,k} \right] \Delta t$
    \State Set objective to $\max(\text{Profit})$.
    
    \State \textbf{Add Constraints} for $i \in \text{CS}, k \in [0, H-1]$:
    \If{$k=0$}
        \State $E_{i,k} = E_{\text{initial},i} + (\eta_{\text{ch}} P^{\text{ch}}_{i,k} - \frac{1}{\eta_{\text{dis}}} P^{\text{dis}}_{i,k}) \cdot \Delta t$
    \Else
        \State $E_{i,k} = E_{i,k-1} + (\eta_{\text{ch}} P^{\text{ch}}_{i,k} - \frac{1}{\eta_{\text{dis}}} P^{\text{dis}}_{i,k}) \cdot \Delta t$
    \EndIf
    \State $0 \le P^{\text{ch}}_{i,k} \le P^{\text{ch,max}}_{i} \cdot z_{i,k}$
    \State $0 \le P^{\text{dis}}_{i,k} \le P^{\text{dis,max}}_{i} \cdot (1 - z_{i,k})$
    \State $E^{\text{min}}_{i} \le E_{i,k} \le E^{\text{max}}_{i}$
    \If{$k = k_{\text{dep},i}$}
        \State $E_{i,k} \ge E^{\text{des}}_{i}$
    \EndIf
    \State $\sum_{i \in \text{CS}} (P^{\text{ch}}_{i,k} - P^{\text{dis}}_{i,k}) \le P^{\text{tr,max}}$
    
    \State \textbf{Solve Problem:}
    \State solution $\leftarrow$ \Call{SolveMILP}{Problem}
    
    \If{solution is Optimal}
        \State Extract action plan $\{a_k^*\}_{k=0}^{N_c-1}$ from solution.
        \State \Return action plan
    \Else
        \State \Return empty plan or default action
    \EndIf
\EndFunction
\end{algorithmic}
\end{algorithm}

\subsection{Mathematical Formulation}
The mathematical structure of the optimization problem solved in Algorithm \ref{alg:milp_mpc} is a classic \textbf{Mixed-Integer Linear Program (MILP)}. This classification is justified as follows:
\begin{itemize}
    \item \textbf{Linear Objective Function:} The objective function is a linear combination of the continuous power variables $P^{\text{ch}}$ and $P^{\text{dis}}$.
    \item \textbf{Linear Constraints:} All system constraints, including energy dynamics, power limits, and state of energy bounds, are formulated as linear equations or inequalities.
    \item \textbf{Mixed-Integer Variables:} The formulation employs both continuous variables (e.g., $P^{\text{ch}}_{i,k}$, $E_{i,k}$) and discrete, binary integer variables ($z_{i,k}$). The binary variables are essential for modeling the logical decision to either charge or discharge at any given time step.
\end{itemize}

\section{Quadratic MPC Formulation (CVXPY Implementation)}

As an alternative to the purely linear model, a quadratic formulation is also implemented. This version extends the previous MILP by introducing quadratic penalty terms into the objective function. While it shares the same underlying constraint structure, the change in the objective function alters the problem's nature to a \textbf{Mixed-Integer Quadratic Program (MIQP)}. The control logic is detailed in Algorithm \ref{alg:miqp_mpc}.

\paragraph{Implementation Details}
This controller is built using \textbf{CVXPY}, a Python-embedded modeling language for convex optimization problems. Due to the presence of both quadratic terms and integer variables, this formulation requires a solver capable of handling MIQPs, such as SCIP, Gurobi, or MOSEK.

\begin{algorithm}[H]
\caption{Online Model Predictive Control (MIQP)}
\label{alg:miqp_mpc}
\begin{algorithmic}[1]
\Function{GetAction}{$E_{\text{initial}}, k_{\text{dep}}, c_{\text{buy}}, c_{\text{sell}}, H, N_c$}
    \State \textbf{Inputs:} Same as Algorithm \ref{alg:milp_mpc}.
    
    \State \textbf{Initialize Problem:} Create a new MIQP problem.
    
    \State \textbf{Define Variables:} Same as Algorithm \ref{alg:milp_mpc}.
    
    \State \textbf{Define Objective Function:}
    \State LinearProfit $\leftarrow \sum_{k=0}^{H-1} \sum_{i \in \text{CS}} \left[ (c^{\text{user}} - c^{\text{buy}}_k - c^{\text{deg}}) P^{\text{ch}}_{i,k} + (c^{\text{sell}}_k - c^{\text{deg}}) P^{\text{dis}}_{i,k} \right] \Delta t$
    \State QuadraticPenalty $\leftarrow \sum_{k=0}^{H-1} \sum_{i \in \text{CS}} \left[ \lambda_{\text{ch}} (P^{\text{ch}}_{i,k})^2 + \lambda_{\text{dis}} (P^{\text{dis}}_{i,k})^2 \right] \Delta t$
    \State Set objective to $\max(\text{LinearProfit} - \text{QuadraticPenalty})$.
    
    \State \textbf{Add Constraints:} Same as Algorithm \ref{alg:milp_mpc}.
    
    \State \textbf{Solve Problem:}
    \State solution $\leftarrow$ \Call{SolveMIQP}{Problem}
    
    \If{solution is Optimal}
        \State Extract action plan $\{a_k^*\}_{k=0}^{N_c-1}$ from solution.
        \State \Return action plan
    \Else
        \State \Return empty plan or default action
    \EndIf
\EndFunction
\end{algorithmic}
\end{algorithm}

The motivation for the quadratic penalty terms is to encourage smoother control actions. By making very high power flows quadratically more "expensive," the optimizer is incentivized to find solutions that are less aggressive, which can be beneficial for battery health and for reducing stress on the local grid infrastructure.

\section{Lyapunov-based Adaptive Horizon MPC}

A key enhancement developed in this work is the \textbf{Lyapunov-based Adaptive Horizon MPC}, which aims to reduce the computational burden of the online MPC while retaining its optimality and stability guarantees. The controller dynamically adjusts its prediction horizon $H_t$ based on the stability of the system, which is formally assessed using a Lyapunov function. A Lyapunov function $V(x)$ is a scalar function that measures the system's deviation from a desired equilibrium state. For the V2G system, we define it as the sum of squared errors from the desired final energy state:
\begin{equation}
    V(E_t) = \sum_{i \in \text{EVs}} (E_{i,t} - E_i^{\text{des}})^2
\end{equation}
The adaptive control logic, which wraps either the MILP or MIQP solver, is detailed in Algorithm \ref{alg:lyapunov_mpc}.

\begin{algorithm}[H]
\caption{Lyapunov-based Adaptive Horizon MPC}
\label{alg:lyapunov_mpc}
\begin{algorithmic}[1]
\State \textbf{Initialization:}
\State Initialize current horizon $H_{\text{current}} \leftarrow H_{\text{max}}$.
\State Define horizon bounds $H_{\text{min}}, H_{\text{max}}$ and convergence rate $\alpha$.
\State Initialize action plan $\mathcal{A} \leftarrow \emptyset$.

\Function{GetAdaptiveAction}{current state $s_t$}
    \If{action plan $\mathcal{A}$ is not empty}
        \State $a_t \leftarrow \text{Pop first action from } \mathcal{A}$.
        \State \Return $a_t$.
    \EndIf
    
    \State \Comment{Plan is empty, re-optimization is needed}
    \State $E_t \leftarrow \text{Get current energy states from } s_t$.
    \State $V(E_t) \leftarrow \sum_{i} (E_{i,t} - E_i^{\text{des}})^2$.
    
    \State \Comment{Solve MPC with the current horizon}
    \State solution $\leftarrow$ \Call{SolveMPC}{$s_t, H_{\text{current}}$} \Comment{Using MILP or MIQP solver}
    
    \If{solution is Optimal}
        \State Extract optimal first action $a_t^* = (P^{\text{ch,*}}_{t}, P^{\text{dis,*}}_{t})$.
        \State Predict next energy state $E_{t+1}$ using $a_t^*$.
        \State $V(E_{t+1}) \leftarrow \sum_{i} (E_{i,t+1} - E_i^{\text{des}})^2$.
        
        \State \Comment{Verify Lyapunov stability condition}
        \If{$V(E_{t+1}) \le V(E_t) - \alpha V(E_t)$}
            \State \Comment{Stable: reduce computational load for next cycle}
            \State $H_{\text{next}} \leftarrow \max(H_{\text{min}}, H_{\text{current}} - 1)$.
        \Else
            \State \Comment{Not stable enough: increase planning depth}
            \State $H_{\text{next}} \leftarrow \min(H_{\text{max}}, H_{\text{current}} + 1)$.
        \EndIf
        \State Extract new action plan $\mathcal{A}$ from solution.
    \Else
        \State \Comment{Solver failed: increase horizon as a safeguard}
        \State $H_{\text{next}} \leftarrow \min(H_{\text{max}}, H_{\text{current}} + 1)$.
        \State $\mathcal{A} \leftarrow \text{default safe action}$.
    \EndIf
    
    \State $H_{\text{current}} \leftarrow H_{\text{next}}$.
    \State $a_t \leftarrow \text{Pop first action from } \mathcal{A}$.
    \State \Return $a_t$.
\EndFunction
\end{algorithmic}
\end{algorithm}

This intelligent adjustment makes the online MPC more efficient and practical, reducing computation time during stable periods while retaining the ability to perform deep planning when necessary to guarantee system stability and constraint satisfaction.



%%%%%%%%%%%%%%%
\section{Approximate Explicit MPC: A Machine Learning Approach}
The online, implicit MPC formulation provides high-quality control decisions by solving an optimization problem at every time step. However, this approach has a significant drawback: its computational complexity. For scenarios with a large number of EVs or a long control horizon, solving a Mixed-Integer Linear Program (MILP) in real-time can be prohibitively slow, making it impractical for many real-world applications.
\noindent
To overcome this limitation, this work implements an \textbf{Approximate Explicit Model Predictive Controller (A-MPC)}. This controller leverages machine learning to replace the computationally expensive online optimization with a fast, lightweight inference step.

\subsection{Methodology: From Oracle to Apprentice}
The core idea is to treat the slow but powerful online MPC as an "oracle" or expert teacher. An apprentice model, $f_{\theta}$, is trained to mimic the oracle's behavior. The process involves an offline training phase to generate the model, followed by a fast online inference phase for real-time control. The state $s_t$ used for this mapping is a fixed-size vector summarizing all necessary information for a control decision:
\begin{equation}
    s_t = [ \mathbf{SoC}, \mathbf{T}^{\text{rem}}, \mathbf{C}^{\text{ch}}, \mathbf{C}^{\text{dis}} ]^T
\end{equation}
where $\mathbf{SoC}$ is the vector of current States of Charge, $\mathbf{T}^{\text{rem}}$ is the vector of remaining times until departure, and $\mathbf{C}^{\text{ch}}, \mathbf{C}^{\text{dis}}$ are the vectors of predicted future electricity prices over the horizon $H$. All vectors are padded to a maximum size to ensure a consistent input dimension for the model.

\subsection{Approximation via Random Forest}
The first implementation of the A-MPC uses a Random Forest, a powerful ensemble learning method known for its robustness and good performance on tabular data without extensive hyperparameter tuning.

\paragraph{Offline Training}
The training process, detailed in Algorithm \ref{alg:ampc_training_rf}, involves generating a large dataset by repeatedly querying the MPC oracle across many diverse scenarios and system states. A \texttt{RandomForestRegressor} model is then trained on this dataset using a standard fitting procedure.

\begin{algorithm}[H]
\caption{Offline Training of Random Forest MPC Approximator}
\label{alg:ampc_training_rf}
\begin{algorithmic}[1]
\State \textbf{Inputs:} Number of samples $N_{samples}$, Set of scenarios $\mathcal{S}$, MPC Oracle $O_{MPC}$.
\State \textbf{Initialize:} Empty datasets $X \leftarrow \emptyset$, $Y \leftarrow \emptyset$.

\For{$i = 1$ to $N_{samples}$} \Comment{Data Generation Loop}
    \State Randomly select a scenario $s_{conf} \in \mathcal{S}$ and initialize environment $env$.
    \State Select a random timestep $t_{rand}$ and advance $env$ to that state.
    \State Construct state vector $s_t \leftarrow$ \Call{BuildStateVector}{$env$}.
    \State Obtain optimal action from oracle: $a_t^* \leftarrow O_{MPC}(s_t)$.
    \If{$a_t^*$ is a valid, non-trivial action}
        \State Append $s_t$ to dataset $X$; Append $a_t^*$ to dataset $Y$.
    \EndIf
\EndFor

\State \Comment{Model Training}
\State Initialize model $f_{\theta} \leftarrow \text{RandomForestRegressor}(\text{hyperparameters})$.
\State Train the model on the entire dataset: $f_{\theta}.\text{fit}(X, Y)$.
\State \Return Trained model $f_{\theta}$.
\end{algorithmic}
\end{algorithm}

\paragraph{Implementation Details}
This controller is implemented in the \texttt{ApproximateExplicitMPC} class. The model is generated by the \texttt{train\_mpc\_approximator.py} script, which executes the steps outlined in Algorithm \ref{alg:ampc_training_rf}. The trained \textbf{scikit-learn} model is serialized to \texttt{mpc\_approximator.joblib}.

\subsection{Approximation via Deep ReLU Network}
While the Random Forest provides a powerful general-purpose approximation, a more theoretically grounded approach for this problem is to use a deep neural network with Rectified Linear Unit (ReLU) activation functions. As detailed in Chapter 2, the explicit solution to a linear MPC problem is a Piecewise Affine (PWA) function, and a deep ReLU network is theoretically capable of exactly representing such a function \cite{karg2020efficient}.

\paragraph{Offline Training}
The training methodology for the neural network, detailed in Algorithm \ref{alg:ampc_training_nn}, follows the same oracle-apprentice data generation paradigm. However, the training phase is iterative, involving epochs, mini-batches, and gradient-based optimization to minimize the Mean Squared Error (MSE) between the network's predictions and the oracle's actions.

\begin{algorithm}[H]
\caption{Offline Training of Neural Network MPC Approximator}
\label{alg:ampc_training_nn}
\begin{algorithmic}[1]
\State \textbf{Inputs:} $N_{samples}$, $\mathcal{S}$, $O_{MPC}$, epochs, batch size $B$, learning rate $\eta$.
\State \textbf{Data Generation:}
\State Generate datasets $(X, Y)$ using the same procedure as lines 2-10 in Algorithm \ref{alg:ampc_training_rf}.

\State \Comment{Model Training}
\State Initialize model $f_{\theta} \leftarrow \text{MPCApproximatorNet}(\text{architecture})$.
\State Initialize optimizer (e.g., Adam) with learning rate $\eta$.
\State Initialize loss function $L \leftarrow \text{MSELoss}$.
\State Create DataLoader $D_L$ from $(X, Y)$ with batch size $B$.

\For{epoch = 1 to epochs}
    \For{batch $(s_b, a_b)$ in $D_L$}
        \State Zero gradients: optimizer.zero\_grad().
        \State \Comment{Forward pass}
        \State Predict actions: $\hat{a}_b \leftarrow f_{\theta}(s_b)$.
        \State \Comment{Compute loss and backpropagate}
        \State loss $\leftarrow L(\hat{a}_b, a_b)$.
        \State loss.backward().
        \State \Comment{Update model weights}
        \State optimizer.step().
    \EndFor
\EndFor
\State \Return Trained model $f_{\theta}$.
\end{algorithmic}
\end{algorithm}

\paragraph{Implementation Details}
This controller is implemented as the \texttt{ApproximateExplicitMPC\_NN} class. The \texttt{train\_mpc\_approximator\_nn.py} script executes Algorithm \ref{alg:ampc_training_nn}, parallelizing the data generation process for efficiency. The trained \textbf{PyTorch} model is saved to \texttt{mpc\_approximator\_nn.pth}.

\subsection{Online Inference}
Once either the Random Forest or the Neural Network model is trained, it can be deployed for real-time control. The online inference process, shown in Algorithm \ref{alg:ampc_inference}, is identical for both approximators and is orders of magnitude faster than solving the online MPC problem.

\begin{algorithm}[H]
\caption{Online Inference with Trained A-MPC}
\label{alg:ampc_inference}
\begin{algorithmic}[1]
\State \textbf{Input:} A trained apprentice model $f_{\theta}$ (either RF or NN), current environment state $env_t$.
\Function{GetAction}{$env_t$}
    \State Construct state vector $s_t \leftarrow$ \Call{BuildStateVector}{$env_t$}.
    \State Compute action via fast inference: $a_t \leftarrow f_{\theta}(s_t)$.
    \State \Return $a_t$.
\EndFunction
\end{algorithmic}
\end{algorithm}
