% ===================================================================
% CHAPTER 1: INTRODUCTION
% ===================================================================
\chapter{Introduction}
The global strategy for decarbonizing transport heavily relies on the shift toward electric mobility. This thesis investigates the complex challenges and opportunities arising from the large-scale integration of electric vehicles (EVs), as illustrated in Figure \ref{fig:growth}, into existing power grids.

\begin{figure}[H]
    \centering
    \includegraphics[width=0.8\textwidth]{ExpectedEV.png}
    \caption{Expected growth of EV sales in the coming years (Image from: \footnote{Tavakoli2019})}
    \label{fig:growth}
\end{figure}

\subsection{Background and Relevance of Electric Vehicles and Vehicle-to-Grid}

\begin{center}
  \begin{minipage}{0.8\textwidth}
    \begin{displayquote}
      \large\itshape
      "Electric car sales continue to break records globally, particularly in China and other emerging economies."\\[1em]

    \end{displayquote} 
    
  \end{minipage} 
\end{center}


\begin{flushright}
  \textsc{\textbf{International Energy Agency (IEA)}
\footnote{\href{https://www.iea.org/reports/global-ev-outlook-2025/executive-summary}{Global EV Outlook 2025 - Executive Summary}}}
\end{flushright}
\noindent
The rapidly expanding Electric Vehicle (EV) market is reshaping modern mobility, offering a path to reduced carbon emissions and enhanced energy efficiency \footcite{orfanoudakis2024ev2gym}. This transition is fundamental to environmental sustainability, as it lessens dependence on fossil fuels, mitigates climate change by cutting greenhouse gas emissions, and improves urban air quality. However, integrating millions of EVs into the power system is a significant challenge, threatening to intensify peak demand, strain transmission and distribution networks, and cause technical issues like voltage irregularities or line losses \footcite{orfanoudakis2024ev2gym, salvatti2020electric}.
\\
\noindent
This challenge raises a critical question: \textbf{can we transform this apparent liability into a foundational asset for grid stability?} The answer may be found in the Vehicle-to-Grid (V2G) paradigm. V2G reimagines EVs not as passive loads, but as mobile, flexible energy assets capable of bidirectional power exchange \footcite{alfaverh2022optima}. This potential is significant, considering EVs remain parked for approximately 96\% of the day, providing a vast window for grid interaction \footcite{evertsson2024investigating}. Furthermore, the rapid responsiveness of EV batteries makes them ideal for providing ancillary services like frequency regulation \footcite{alfaverh2022optima}. A growing body of research, reviewed by Qiu et al. \footcite{Qiu2023} and Xie \footcite{Xie2025}, suggests that intelligent, bidirectional charging can offset the negative impacts of EV integration. The central proposition of this thesis is to test this hypothesis: that through advanced control methodologies, V2G can be proven not just theoretically sound, but practically indispensable, provided its economic and grid-stabilizing benefits demonstrably outweigh costs such as battery degradation and infrastructure investment.
\\
\noindent
Alongside V2G, other bidirectional power flow schemes, shown in Figure \ref{fig:schemes}, have been proposed to enhance energy resilience:

\begin{figure}[H]
    \centering
    \includegraphics[width=0.5\textwidth]{Ev_1.png}
    \caption{Other schemes of bidirectional power flow}
    \label{fig:schemes}
\end{figure}

\begin{enumerate}
    \item \textbf{Vehicle-to-Home (V2H):} An EV powers a household during outages or high-cost periods, boosting domestic energy security.
    \item \textbf{Vehicle-to-Building (V2B):} This concept is extended to commercial or industrial facilities, where EVs support load management and optimize energy consumption.
    \item \textbf{Vehicle-to-Vehicle (V2V):} Direct power transfer between EVs provides a solution for emergency charging or resource sharing.
\end{enumerate}
Collectively, these modalities underscore the versatility of EV batteries as distributed energy resources, advancing the transition to a more sustainable energy ecosystem.

\subsection{Challenges in EV Integration into the Electricity Grid and the Role of Artificial Intelligence}
Modern electricity systems face increasing strain from the integration of intermittent \textbf{Renewable Energy Sources (RESs)} like wind and solar. The inherent variability of RESs leads to significant power generation swings, creating supply-demand mismatches that fuel price volatility and complicate grid management. This continuous instability challenges the economic efficiency and reliability of the grid, proving difficult for conventional control frameworks to manage \footcite{orfanoudakis2024ev2gym, minchala2025systematic}.
\\
\noindent
The parallel rise of EV adoption and RES deployment has created an environment of unprecedented uncertainty and complexity. This situation makes traditional, rule-based controllers, designed for a more predictable and centralized grid, increasingly inadequate. \textbf{Is it possible, then, that a new control paradigm is needed?} The literature, as reviewed by NaXu et al. \footcite{NaXu} and Feyijimi et al. \footcite{Feyijimi}, strongly suggests so, highlighting the limitations of legacy systems and framing the problem in a way that points toward data-driven methods. This shift signals a genuine paradigm change towards a \emph{smart grid}, where adaptive, real-time, and autonomous operation is vital \footcite{NaXu}.
\\
\noindent
This has led to a surge in research focused on \textbf{Reinforcement Learning (RL)}, a paradigm that can, in theory, learn optimal control policies directly from environmental interaction without a perfect system model. While meta-heuristic algorithms have been explored, they often lack the real-time adaptability required for dynamic control \footcite{Srihari}. The hypothesis to be tested is whether the theoretical promise of RL holds up against the engineering realities of the V2G problem. From this perspective, RL is not merely an optimization tool but an \textbf{enabling technology} for a more cognitive and robust energy infrastructure capable of navigating a decarbonized future. Within this domain, \textbf{Deep Reinforcement Learning (DRL)} has emerged as a particularly powerful approach, valued for its capacity to derive near-optimal strategies in dynamic and uncertain environments without relying on precise models or forecasts \footcite{orfanoudakis2024ev2gym}.

\subsection{Objectives and Contributions of the Thesis}
This thesis confronts the complex multi-objective optimization problem at the heart of Vehicle-to-Grid (V2G) systems. The overarching objective is to move beyond a purely theoretical analysis by actively developing, testing, and enhancing a high-fidelity simulation architecture. This platform serves as a digital twin to rigorously evaluate and compare advanced control strategies, balancing economic benefits, user mobility needs, battery health, and grid stability under realistic stochastic conditions.
\\
\noindent
More than a simple review of existing literature, this work focuses on the practical implementation and validation of a V2G simulation framework in Python. This tool is leveraged to demonstrate and explore novel perspectives for training intelligent agents. The main contributions are:

\begin{itemize}
    \item \textbf{Enhancement of a V2G Simulation Architecture:} A key contribution is the systematic testing, validation, and enhancement of the \textbf{EV2Gym} simulation framework. This work solidifies its role as a robust platform for benchmarking control algorithms and includes the development of an interactive data application using \textbf{Streamlit} for results visualization and scenario analysis.

    \item \textbf{Development of a Battery Degradation Calibration Algorithm:} A novel algorithm for calibrating the battery degradation model is presented. This contribution ensures that the simulation's battery health predictions are grounded in realistic parameters, increasing the fidelity of the economic and physical assessments of V2G strategies.

    \item \textbf{Exploration of Novel Reinforcement Learning Perspectives:} The validated simulation environment is used to investigate and implement advanced training methodologies for RL agents. A key focus is placed on techniques like \textbf{adaptive reward shaping}, where the reward function dynamically evolves during training to guide the agent towards a more holistic and robust control policy.

    \item \textbf{Practical Implementation and Comparison of Advanced MPC Formulations:} The thesis details the development and implementation of multiple advanced model-based controllers. This includes an \textbf{explicit MPC}, a classic \textbf{implicit MPC}, and a novel \textbf{Adaptive Horizon Model Predictive Control (AHMPC)}, all formulated in PuLP. These are benchmarked against the theoretical offline optimal controller to rigorously analyze the trade-offs inherent in real-world, online deployment with limited future information.
\end{itemize}

\subsection{Research Methodology}
The research was guided by a central question: how can the economic profit of Vehicle-to-Grid (V2G) operations be maximized without imposing undue costs in terms of battery degradation or creating instability by overloading local grid infrastructure? To address this, a systematic methodology was adopted, rooted in the philosophical principle of \textbf{falsifiability} as articulated by Karl Popper \footcite{NielsonElton2021InductionPopperML}. The research is structured to formulate testable propositions that can be rigorously challenged by empirical evidence, where failure is as informative as success.
\\
\noindent
The initial phase involved an extensive literature review using Google Scholar to identify state-of-the-art V2G control strategies and their inherent trade-offs. This was followed by an empirical phase centered on the simulation framework detailed in Chapter 3. The significance of results from various control agents was continuously evaluated through a multi-faceted validation process, including comparative analysis against published benchmarks, heuristic evaluation based on acquired expertise, and grounding empirical findings in the foundational knowledge from the literature review.
\\
\noindent
To further systematize the literature review, a quantitative analysis was performed on the collected papers in the \texttt{Papers\_read} directory using a custom Python script. This analysis produced two key visualizations. The first, a weighted word cloud (Figure \ref{fig:pagerank_wordcloud}), was generated from titles and abstracts, with word size corresponding to its \textbf{PageRank} score within a citation graph.

\begin{figure}[H]
    \centering
    \includegraphics[width=0.8\textwidth]{Page_rank.png}
    \caption{Weighted word cloud generated from the literature review, highlighting key research themes. The size of each word corresponds to its frequency and importance, derived from the PageRank analysis.}
    \label{fig:pagerank_wordcloud}
\end{figure}

\noindent
This visualization confirms the research focus, with primary terms like \textbf{Energy}, \textbf{Power}, \textbf{Grid}, and \textbf{EVs} setting the context. The equal prominence of \textbf{Learning} and \textbf{Control}, alongside strong secondary terms like \textbf{Reinforcement}, \textbf{Agent}, and \textbf{Policy}, firmly anchors the methodology in Reinforcement Learning. The inclusion of \textbf{MPC} (Model Predictive Control) highlights the central dialogue between model-free and model-based approaches explored in this thesis.

\subsection{Thesis Structure}
The remainder of this thesis is organized as follows:
\begin{itemize}
    \item \textbf{Chapter 2: Overview of Optimal Management of EV Charging and Discharging} provides foundational knowledge on V2G technology, the complex multi-objective nature of EV charging optimization, and presents a comprehensive review of state-of-the-art research approaches.

    \item \textbf{Chapter 3: The V2G Simulation Framework: A Digital Twin for V2G Research} details the architecture and core models of the simulation environment. This chapter describes the enhancements made to the framework, establishing it as the central experimental platform for implementing and evaluating the control agents analyzed in this work.

    \item \textbf{Chapter 4: Experimental Campaign and Results Analysis} presents the main results from the comparative analysis of the different control strategies (DRL, MPCs, heuristics). It analyzes the performance of novel training techniques, discusses the implications of the findings, and provides a detailed \textbf{sensitivity analysis} on the most critical system parameters to assess the robustness of the conclusions.

    \item \textbf{Bibliography} lists all cited references.
\end{itemize}